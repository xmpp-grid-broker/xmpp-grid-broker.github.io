% !TeX spellcheck = en_GB

\section{Requirements}\label{sec:requirements}

The following sections describe the primary requirements in the form of user stories~\cite{agile-alliance-user-stories}.
Figure~\ref{fig:requirements-overview} shows an overview of the primary use stories.

\begin{figure}[h]
    \centering
    \includegraphics[width=1\linewidth]{resources/requirements_overview}
    \caption[Use case diagram]{UML use case diagram presenting an overview of the primary user stories.}
    \label{fig:requirements-overview}
\end{figure}

\subsection{Authentication}\label{sec:authentication}
\subsubsection{Login}

As an Administrator,\\
I want to log in\\
- preferably using an existing client \gls{tls} certificate - \\
so that only I can inspect and manage topics.\\

\subsubsection{Secure \gls{xmpp} Authentication}

As an Administrator concerned with security requirements,\\
I want to use either \gls{sasl-external} or \gls{sasl-scram} mechanism for authentication -

\begin{itemize}
    \item preferably the SCRAM-SHA-256-PLUS variant and
    \item preferably using mutual certificate-based authentication including revocation status checking
\end{itemize}

\noindent - so that the controller is fully compatible with the \gls{xmpp-grid-standard}~\cite{ietf-mile-xmpp-grid-05}.

\noindent To achieve this goal, I am willing to accept:
\begin{itemize}
    \item More costly and less user friendly authentication
    \item limited compatibility of supported \gls{xmpp} servers
\end{itemize}

\subsubsection{Secure \gls{xmpp} Connection}

As an Administrator concerned with security requirements,\\
I want to use minimally \gls{tls} 1.2 [RFC5246] to communicate with the \gls{xmpp} server at all times\\
to achieve maximal security and compatibility with the \gls{xmpp-grid-standard}~\cite{ietf-mile-xmpp-grid-05}.

\subsubsection{Secure Connection}

As an Administrator concerned with security requirements,\\
I want to use minimally \gls{tls} 1.2 [RFC5246] to communicate with the \gls{broker}\\
to achieve maximal security.

\subsubsection{Multiple Administrators}\label{sec:requirement-multiple-administrators}

As an Administrator,\\
I want to grant access to administrators \\
so that they can also manage the application.

\subsubsection{Audit Trail}\label{sec:requirement-audit-trail}

As an Administrator concerned with security requirements,\\
I want to be able to access an audit log\\
- preferably using existing \gls{xmpp} mechanisms - \\
so that I can reconstruct what other Administrations did on the controller.

\subsubsection{Logout}\label{sec:requirement-logout}

As an Administrator,\\
I want to log out\\
so that I can terminate a session.

\subsection{List Topics and Collections}\label{sec:list-topics}

\subsubsection{List All Topics}\label{sec:requirement-list-all-topics}
As an Administrator,\\
I want to see a list of all topics of the associated controller\\
so that I can quickly assimilate which topics exist.

\subsubsection{List All Top-Level-Collections}
As an Administrator,\\
I want to see a list of all top-level-collections of the associated controller\\
so that I can quickly assimilate which collections exist.

\subsubsection{List All Parent-Collections of a Topic}
As an Administrator,\\
I want to see a list of all transitive parent collections that contain a given topic\\
so that I can quickly assimilate in which collections items are published.

\subsubsection{List All Subtopics and Subcollection of a Collection}
As an Administrator,\\
I want to see a list of all collections and topics that a given collection contains\\
so that I can quickly assimilate the collection hierarchy.

\subsubsection{List Available topics With Limited Access (optional)}

As an Administrator,\\
I want to see a list of all topics of the associated controller to which I have limited access to,\\
to simplify troubleshooting and locate errors.

\subsubsection{List Available collections With Limited Access (optional)}

As an Administrator,\\
I want to see a list of all collections of the associated controller to which I have limited access to,\\
to simplify troubleshooting and locate errors.

\subsubsection{Topic and Collection Paging}
As an Administrator,\\
I want to be able to page through any set of collection/topic with more than 10 Items \\
so that I can work with more than 1000 collections and topics more effectively.

\subsubsection{Topic and Collection Name Filter}\label{sec:requirement-topic-filter}
As an Administrator,\\
I want to be able to quickly filter any set of collections/topics with more than 10 Items \\
so that I can work with more than 1000 collections and topics more effectively.

\subsection{Create a New Topic}\label{sec:create-topic}

As an Administrator,\\
I want to create a new topic on the associated controller\\
so that I am not tied to a fixed set of topics.

\subsection{Create a New Collection}\label{sec:create-collection}

As an Administrator,\\
I want to create a new collection on the associated controller\\
so that I can flexibly patch topics together.

\subsubsection{Override Default Topic Configuration}\label{sec:requirement-topic-default-configuration}

As an Administrator in the process of creating a new topic,\\
I want to override the default configuration (e.g. the affiliations) \\
so that I can restrict access and provide reasonable defaults.

\subsubsection{Override Default Collection Configuration}\label{sec:requirement-collection-default-configuration}

As an Administrator in the process of creating a new collection,\\
I want to override the default configuration (e.g. the affiliations) \\
so that I can restrict access and provide reasonable defaults.

\subsubsection{Initial topic Consumers and Providers}\label{sec:requirement-initial-topic-consumer-provider}

As an Administrator in the process of creating a new topic,\\
I want to specify an initial set of consumers and providers \\
so that I can restrict access to that topic and provide reasonable defaults.

\subsubsection{Initial Collection Consumers}\label{sec:requirement-initial-collection-consumer}

As an Administrator in the process of creating a new collection,\\
I want to specify an initial set of consumers \\
so that I can restrict access to that collection and provide reasonable defaults.

\subsection{Delete an Existing Topic}\label{sec:delete-topic}

As an Administrator,\\
I want to delete an existing topic on the associated controller\\
so that I can get rid of obsolete topics.

\subsection{Delete an Existing Collection}\label{sec:delete-collection}

As an Administrator,\\
I want to delete an existing collection on the associated controller\\
so that I can get rid of obsolete collections.


\subsubsection{Fault Prevention On Topic-Delete}

As an Administrator in the process of deleting a topic, \\
I want a mechanism to prevent me from deleting the wrong topic on the associated controller\\
(e.g. require me to enter the name of the topic manually).

\subsubsection{Fault Prevention On Collection-Delete}

As an Administrator in the process of deleting a collection, \\
I want a mechanism to prevent me from deleting the wrong collection on the associated controller\\
(e.g. require me to enter the name of the collection manually).


\subsection{Manage Topic/Collection Subscriptions}\label{sec:manage-subscriptions}

\subsubsection{List Consumers}

As an Administrator, \\
I want to list all consumers (including their JIDs) of a given topic/collection on the associated controller, \\
so that I can verify that specific consumers are subscribed, and others are not.


\subsubsection{Inspect Detailed Subscription Configuration}

As an Administrator, \\
I want to inspect the detailed topic/collection subscription configuration of a given consumer, \\
so that I can reproduce and reason about the receipt of data on that consumer
and find potential misconfiguration.

\subsubsection{Partially Modify Subscription Configuration}

As an Administrator, \\
I want to modify parts of the topic/collection subscription configuration of a given consumer, \\
so that I can fix misconfiguration.

\subsubsection{Unsubscribe Consumer}

As an Administrator, \\
I want to manually unsubscribe a specific consumer from a particular topic/collection on the associated controller, \\
so that I can remove obsolete or undesired subscriptions.

\subsubsection{Subscribe Consumer}

As an Administrator, \\
I want to manually subscribe a specific consumer on a particular topic/collection on the associated controller, \\
so that I can faster setup and manage consumers.

\subsection{Manage Topic Affiliations}\label{sec:manage-affiliations}
\subsubsection{Inspect Affiliations}

As an Administrator,\\
I want to list all Affiliations (JID and "Role") for a particular topic/collection on the associated controller \\
so that I can find potential misconfiguration.

\subsubsection{Modify Affiliations}

As an Administrator,\\
I want to modify the Affiliation ("Role") of a given JID for a particular topic/collection on the associated controller \\
so that I can fix potential misconfiguration.

\subsubsection{Fault Prevention When Modifying My Affiliation}

As an Administrator in the process of modifying my Affiliation for a particular topic/collection on the associated controller,\\
I want a mechanism to prevent me from accidentally downgrading my rights.

\subsubsection{Meaningful Error For Topics/Collection With Limited Access}

As an Administrator,\\
I want to receive a meaningful error message when inspecting a topic/collection to which I have limited access \\
so that I can quickly comprehend why the configuration options are limited.

\subsection{Manage Persisted Items of a Topic}\label{sec:manage-persisted-items}
\subsubsection{Inspect Persisted Items}

As an Administrator,\\
I want to list all persisted items for a particular topic on the associated controller \\
so that I can get an overview and check for misconfiguration.

\subsubsection{Filter Persisted Items}\label{sec:requirement-filter-persisted-items}

As an Administrator,\\
I want to be able to filter all persisted items of a specific topic by \\
\begin{itemize}
    \item the timestamp of its publication
    \item the publishers JID
\end{itemize}
so that I can work with more than 10000 persisted items more effectively.

\subsubsection{Paged Persisted Items}\label{sec:paged-persisted-items}
As an Administrator working with filtered persisted items,\\
I want to be able to page through the resulting items\\
- given that this feature is supported by the associated controller -\\
so that I can work with more than 10000 persisted items more effectively.

\subsubsection{Delete a Persisted Items From a Topic}

As an Administrator,\\
I want to delete a particular persisted item from a specific topic\\
- given that this feature is supported by the associated controller -\\
so that I can clean up test items and remove obsolete or corrupted items.

\subsubsection{Purge All Persisted Items From a Topic}

As an Administrator,\\
I want to purge persisted items from a specific topic\\
- given that this feature is supported by the associated controller -\\
so that I can clean up test items and remove obsolete or corrupted items.

\subsubsection{Delete Set of Persisted Items From a Topic (optional)}

As an Administrator,\\
I want to delete a set of persisted item that match a given criteria from a specific topic\\
- given that this feature is supported by the associated controller -\\
so that I can clean up test items and remove obsolete or corrupted items.

\subsection{Manage Subscription Requests (optional)}\label{sec:subscription-requests}

\subsubsection{List Subscription Request}
As an Administrator,\\
I want to list pending subscription requests for a given topic\\
- given that this feature is supported by the associated controller -\\
so that I can quickly assimilate pending requests.

\subsubsection{Accept Subscription Request}

As an Administrator,\\
I want to accept a pending subscription request for a given topic\\
- given that this feature is supported by the associated controller -\\
to enable more dynamic access models than just maintaining a black- or whitelist.

\subsubsection{Reject Subscription Request}

As an Administrator,\\
I want to reject a pending subscription request for a given topic\\
- given that this feature is supported by the associated controller -\\
so that I can deny user access in accordance with the \gls{xmpp} standards.

\subsection{Validate Controller Configuration (optional)}\label{sec:validate-controller-config}

\subsubsection{Validate Supported XEPs Configurations}
As an Administrator,\\
I want to validate that a minimum set of XEPs are supported by the associated controller\\
so that I can quickly identify incompatibilities.

\subsubsection{Validate Optional XEP Implementations}
As an Administrator,\\
I want to validate that the required features that are marked as optional or recommended in the XEPs are implemented by the associated controller\\
so that I can quickly identify incompatibilities.

